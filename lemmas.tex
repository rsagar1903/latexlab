\documentclass{article}
\usepackage[english]{babel}
\newtheorem{theorem}{Theorem}[section]
\newtheorem{corollary}{Corollary}[theorem]
\newtheorem{lemma}[theorem]{Lemma}

\begin{document}
\section{Introduction}
Theorems can be easily defined:

\begin{theorem}
Let \(f\) be a function whose derivative exists in every point,then \(f\) is a continuous function.
\end{theorem}
\begin{theorem}[Pythagorean theorem]
\label{pythagorean}
This is a theorem about right triangles and can be summarized in the next equation
\[x^2+y^2=z^2 \]
\end{theorem}
And a sequence of theorem \ref{pythagorean} is the statement in the next corollary.
\begin{corollary}
There's no right rectangle whose sides  measure 3cm,4cm and 6cm.
\end{corollary}
You can reference theorems such as \ref{pythagorean} when a label is assigned.
\begin{lemma}
Given two line segments whose lenghts are \(a\) and \(b\) resspectively there is a real number \(r\) such that \(b=ra\).
\end{lemma}
\end{document}